\documentclass[a4paper,12pt, titlepage]{article}
%\usepackage{color}
%\definecolor{light-gray}{gray}{0.95}

\usepackage{xcolor}
\usepackage{alltt}
\usepackage{url}
\usepackage{tikz}
\usetikzlibrary{trees}

\definecolor{light-gray}{gray}{0.95}
% Compensate for fbox sep:
\newcommand\Hi[2][light-gray]{%
  \hspace*{-\fboxsep}%
  \colorbox{#1}{#2}%
  \hspace*{-\fboxsep}%
}

\title{Technical Report}

\begin{document}
\maketitle

\tableofcontents


\section{Introduction}

\section{Background}
Orbiscom Ireland, acquired by Mastercard in 2009, develop a product called InControl. The acquisition allowed Mastercard to incorporate InControl into their Value Added Services, a range of products that tie in closely with their core business, the processing of electronic payments. Until 2010 Mastercard Ireland was still primarily concerned with the development of InControl. A division of Mastercard Labs was setup within Mastercard Ireland in 2011. As Mastercard continues to grow, more and more departments are being given a presence within Ireland. This report documents the activities and projects undertaken while working for InControl. Since their acquisition Orbiscom have become completely integrated into Mastercard and have ceased operating as Orbiscom entirely. InControl has been split into two different products. InControl Direct exists to support Orbiscoms original customers as is managed and maintained entirely within the InControl department. The second product, InControl is an adapted version of the original product. Where InControl Direct is deployed to and hosted by banks, InControl is hosted on BankNet, Mastercards global payments network. In order to explain the services offered by the InControl platform, it is useful to explain how the credit card payment process works.

\subsection{The four party model}
The credit card payment scheme employed by Mastercard involves four separate parties, for this reason it is referred to as the four party model. There are alternate models in use, but Mastercard employs the four party model as it allows the issuing of payment cards to be handled by separate financial institutions. This leaves less overhead for Mastercard and also insures interoperability between the various financial institutions. The credit card holder typically initiates the payment and is represented by a credit card issuing bank, or an issuer. The merchant involved in the payment is represented by an acquiring bank, or an acquirer. Each successful payment goes through three stages; Authorization, Clearing and Settlement.
\begin{itemize}
\item Authorization: The card-holder submits their payment card details to the merchant. The merchants bank, the acquirer, sends a request to Mastercard to identify the card-holders issuing bank. Once this has been determined and the card has been verified the payment is forwarded, by Mastercard, to the issuing bank. It is worth noting that this is the stage in the process where the InControl service is applied if needed. The card-holders bank approves the purchase and blocks the funds in the card-holders account. No money has been transfered from the card-holders account, the money has merely become unusable by the card-holder. The issuing bank forwards the approval to Mastercard, who forwards it to the acquirer who in turn forwards it to the merchant. The payment has now been approved. 
\item Clearing: Sometime after the payment has been authorized, usually at the end of the week, the merchant submits all of their authorized payments to clearing. This is a batch process that occurs at certain set times. After a payment has been cleared the funds have effectively been transfered.
\item Settlement: Come back too, concerns the actual transfer of funds.
\end{itemize}
To facilitate this, a fee is applied to each transfer. Interchange is typically charged by the issuer to the acquirer. This is also where Mastercard makes money from each payment. The rate of interchange varies from bank to bank and can even be different depending on the nature of the purchase. In order to sustain credit card usage and adoption the services offered by Mastercard must outweigh this additional fee. This report will make no attempts to the discuss politics surrounding this fee and will simply assume the following. (REMOVE: sometimes I love academia ) The interchange fee is to some extent ultimately payed by the merchant. This means that they lose an amount on every purchase made with a payment card as opposed to cash. In order for merchants to continue to accept card payments there must be sufficient consumer pressure for acceptance by the merchants. This is done by incentivising consumers with additional benefits not available through cash payment. Some of these incentives are part of Mastercards core network and come as benefits of using an electronic system such as added security, access to credit and accountability. Mastercards Value Added Services range is a set of products designed to add additional benefits for consumers.
\cite{Something about 4 party, also check I have it the right way around http://tinyurl.com/7bbdq4
 }

\subsection{InControl}
The InControl platform allows card-holders to create virtual payment cards. The virtual cards are linked back to the original real card and account and can be used exactly like a normal payment card. At the core of the system is the algorithm used to create virtual cards, however similar systems have also existed. The unique feature offered by InControl is that no changes to existing infrastructure are required. If a payment requires InControl to continue processing then the payment is routed to the InControl platform. This happens entirely within the payment network. Neither the merchant nor the acquiring bank has any knowledge that InControl has been applied, so no action is needed on their part. Other similar technologies required the merchant to add knowledge of the process to their point of sale, ie, they had to replace all of their card readers.



MasterCard operates Banknet, a global telecommunications network linking all MasterCard card issuers, acquirers and data processing centers into a single financial network. The operations hub is located in St. Louis, Missouri. Banknet uses the ISO 8583 protocol. The network is peer-to-peer mesh network with a set of endpoints. At no time during my internship did I have any involvement with any aspect of Banknet, but it does have one very notable effect on development within the company, the release cycles. Banknet is the core of Mastercards business. One of the most important aspects of an electronic payments service is reliability. If Mastercard is unable to serve it's customers in anyway it would mean a huge blow for their brand, a brand they have invested a substantial amount of money in. Banknet must be reliable. This is achieved in part with an extremely conservative attitude towards updating the software that controls the network. The servers can only be restarted twice a year. This takes a considerable amount of effort. Each node must be updated and "fliped" across the entire network. The network cannot be taken down completely during this time and their are many possible issues that are taken into account, hence the twice yearly cycle. There are a further two periods each year where the network can be updated without restarting the core services. All of this means Mastercard uses a quarterly release cycle, with fixed deadlines. Due to the inflexibility of the release cycle, great care is given to selecting new functionality to be included in each quarters release. These factors, a rigid deadline, fixed requirements and a very strong reliance of reliability in the end product are traits commonly associated with the waterfall design process, and this is the design process employed by Mastercard.

The waterfall design methodology arguably fits well with the requirements surrounding Mastercards core network. Other services, such as InControl are not subject to the same requirements but are non the less hampered (REMOVE) by the employment of waterfall. During my time at Mastercard a switch an agile design process was beginning to get underway within InControl.

\section{Existing System}


\section{Development Process}

\section{SMART}

\section{Conclusion}

\section{References}

\section{Appendices}


\end{document}